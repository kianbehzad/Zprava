
\documentclass[12pt,onecolumn,a4paper]{article}
\usepackage{graphicx}
\usepackage{enumerate}
\usepackage{fancyhdr}
\usepackage{minted}
\usepackage{lastpage}
\usepackage{tcolorbox}
\usepackage{amsmath}
\usepackage[colorlinks=true]{hyperref}
\usepackage{setspace}
\usepackage[absolute]{textpos}    
\usepackage{xepersian}
\usepackage{graphicx}
\settextfont[Scale=1.2]{XW Zar}
\setlatintextfont[Scale=1]{Times New Roman}
\linespread{1.5}
\defpersianfont\nastaliqfont{IranNastaliq}



\begin{document}




\title{گزارش پیشرفت 30 درصد پروژه} 
\author{کیان بهزاد : 9523017\\
پارسا اسدی : 9523005\\
امیر بیات : 9523018}
\date{9 خرداد 1397}
\maketitle

\section{مقدمه} 
در بخش 30 درصد ابتدایی پروژه موارد زیر مد نظر بوده است:
\begin{itemize}
	\item ساخت صفحه
\lr{Login}
و
\lr{Signup}
برای ثبت نام کاربر و وارد شدن کاربر به صفحه اصلی پیامرسان و پیاده سازی بخش سرور آن به صورت کامل و پایدار.
	\item در نظر گرفتن امکان فراموشی کلمه عبور توسط کاربر
	\item اضافه کردن امکان به به خاطر سپردن کلمه عبور کاربر توسط برنامه (مانند سایر پیامرسان ها)
	\item پیاده سازی بخش
\lr{GUI}
صفحه تبادل پیام (به صورت آزمایشی و ابتدایی).
\item امکان ارسال کد تاییدیه به ایمیل کاربر هنگام ثبت نام کاربر 
\end{itemize}
\section{بخش کاربری} 
در این قسمت به بررسی بخشی از پروژه که طرف کاربر است می پردازیم.
\subsection{کلاس \lr{Zprava}}
کلاس
\lr{Zprava}
 که از کلاس
 \lr{QMainWindow}
 ارث بری می کند و صفحه اصلی پروژه را می سازد.\\
در این کلاس یک اشاره گر وجود دارد که به شي از کلاس
 \lr{ZpForm}
 اشاره می کند. \\
در اینجا مختصری از
\lr{ ZpForm}
 توضیح داده می شود و در ادامه بیشتر تشریح می شود.\\
\lr{ZpForm}
 کلاسی است برای طراحی المان های قسمت
\lr{Signup}
و
\lr{Signin}.
در نتیجه اشاره گری به
 \lr{ZpFrom}
 داخل کلاس صفحه اصلی پروژه یعنی
\lr{ Zprava}
 تعریف کردیم تا در نهایت به صفحه اصلی افزوده شود و به نمایش گذاشته شود.\\
داخل کانستراکتور  کلاس
\lr{ Zprava}
 ابتدا ابعاد صفحه داده شده اند. و سپس المان ها به صفحه اصلی افزوده شده اند. 
نکته ای که در اینجا وجود دارد این است که برای اضافه کردن المان ها به صفحه ابتدا
\lr{QWidget}
 را در یک
\lr{ Qlayout}
 قرار می دهیک و آنرا درنهایت به یک
\lr{ QWidget}
 تبدیل میکنیم. این کار دسترسی بهتری به المان های صفحه را نتیجه می دهد.\\
در اینجا نیز به همین شکل این کار انجام شده است. یعنی ابتدا
\lr{ form}
 از نوع
\lr{ ZpForm}
 است که
\lr{ ZpForm}
 از کلاس
\lr{ QWidget}
 ارث بری می کند. سپس به
\lr{ Qlayout}
 اضافه شده است و در نهایت این
 \lr{Qlayout}
 به
\lr{ QWidget}
 که نام آن
\lr{w}
 است اضافه گشته و با تابع
\lr{ setCentralWidget}
 به صفحه اصلی اضافه شده است.


\subsection{کلاس\lr{ ZpForm }}
\lr{Parent}
این کلاس
 \lr{QWidget}
 است. زیرا در کلاس
\lr{ Zprava}
 در تابع
\lr{ setCentralWidget}
 باید یک
\lr{ QWidget}
 ورودی دهیم. در نیتجه این کلاس از
\lr{ QWidget}
 ارث بری می کند. \\
کلاس دارای چند دسته از
\lr{ method}
ها و
\lr{ feature}
 ها است که وظایف متعددی بر عهده دارند. ساخت لمان های داخلی کلاس و طراحی آنها، ارسال اطلاعات در شبکه، ساخت صفحه
\lr{ verify}
 و ساخت
\lr{ button}
 ها و تعریف کارکرد آن ها از جمله این وظایف است.\\
در ادامه توابع تعریف شده در کلاس را مورد بررسی قرار میدهیم.\\
\begin{itemize}
\item \lr{Constructor} : \\
در این تابع ورودی بولی برای این است که اگر کاربر قبلا وارد شده باشد و گزینه
\lr{ remember me} 
 را فعال کرده باشد برای دفعات بعدی مستقیم وارد محیط پیامرسان شود. این کار توسط تابع
\lr{ is\_kept\_logged\_in}
انجام می پذیرد.
.\\
\item \lr{ apply\_stylesheet}:
\\
این تابع فایل
\lr{Qss}
را که برای
\lr{Styling}
برنامه طراحی شده است را به کد اضافه می کند.
\\
\item \lr{create\_form\_widget}:
\\
 در این تابع المان های صفحه به ترتیب گفته شده در بالا ساخته و پرداخته می شوند.
یک نمونه را مورد بررسی قرار می دهیم، 
مثلا در بخش
\lr{ Signup}
 ابتدا آیکون بالایی از نوع
\lr{ QLabel}
که نوعی
\lr {Qwidget}
 است ساخته می شود. سپس اسم و فایل آن به برنامه معرفی می شود. 
سپس
\lr{ layout}
 ساخته می شود و چون المان های بخش ثبت نام به صورت افقی و زیر هم تعریف می شوند درنتیجه از
\lr{ QHBoxLayout} 
 استفاده کردیم. و تغییراتی در آن دادیم و سپس آنرا به
\lr{ QWidget}
 تبدیل کردیم.
برای بقیه نیز همین مراحل طی شده است. در نهایت کل بخش
 \lr{Signup}
 به همین ترتیب پرداخته شده است و در نهایت تمام بخش ورود و ثبت نام هم به همین ترتیب اعمال ساخته شده اند.\\
\item \lr{ create\_verify\_widget}:
\\
 این تابع صفحه
 \lr{Verify}
  را می سازد و که آن المان های صفحه طراحی شده اند و سیگنالی برای عمل کلیک کردن بر روی
\lr{ٰVerify Button}
ساخته شده است که تابع
\lr{ slotVerify\_Button\_Clicked}
 را فراخوانی می کند.
این صفحه وقتی ظاهر می شود که بر دکمه ی
 \lr{Signup}
 کلیک کنیم. در نتیجه ی این کار ایمیلی حاوی کد فعالسازی به کاربر ارسال می شود.\\
\item \lr{create\_forget\_widget}:
\\
این تابع مانند تابع قبلی که توضیح داده شد کار میکند و تفاوت اساسی با آن ندارد.
\lr{ forget}
.
\item \lr{ slotFading\_\#\_widget} :
\\
این تابع به تعدد برای موارد مختلف استفاده شده است (برای همین از 
 \lr{\#}
استفاده شده است) وقتی در برنامه قرار است صفحه جدیدی باز شود از این تابع استفاده میکنیم و با استفاده از
\lr{ fade out}
 کمی انیمیشن به پنجره ها اضافه می کنیم تا زیبا تر شود.
\item \lr{initiate\_networking}:
\\
در این تابع برای تبادل اطلاعات با سرور از کلاس 
\lr{QNetwrokAccessManager}
 استفاده شده است، واطلاعات از طریق پروتوکل
\lr{ http}
 و متد
\lr{ get}
مبادله می شود.
\\
\item \lr{ handle\_reply} :
\\
 این تابع با توجه با
\lr{ State}
کاربر و پاسخ سرور, 
\lr{ QAction}
 ه ارا مقداری تغییر می دهد. مثلا وقتی
 lr{Username}
  از قبل در سرور موجود باشد علامت خطا در صفحه ظاهر می شود.
\item \lr{slotReadyRead} :
\\
سرور برای ما اطلاعات را ارسال میکند هر وقت این اطلاعات قابل خواندن شدند این تابع این اطلاعات را دریافت میکند.\\
\end{itemize}
\section{بخش سرور}
ابتدا تصمیم تیم بر آن بود تا از 
\lr{node.js}
به عنوان زبان برنامه نویسی سرور استفاده کند. در نتیجه در ابتدا قسمت
\lr{Login}
و 
\lr{Signup}
بدین صورت پی ریزی شد و 
\lr{Database}
آن توسط فایل های 
\lr{Json}
نوشته شد.
\\
 اما بعد از نوشته شدن 
\lr{Interface}
در سمت کاربر و تحقیقات بیشتر در مورد سرور تیم به این نتیجه رسید که 
\lr{Django}
نیاز های ما را بهتر برطرف می کند. پس با استفاده از این 
\lr{Framework}
و 
\lr{SQLlite3}
به عنوان 
\lr{Database}
سرور
\lr{Zprava}
بازنویسی شد.
\\
حال در زیر به بررسی نحوه عملکرد سرور میپردازیم.
\begin{enumerate}
\item 
\lr{\textbf{Signup Application}}
\\
این برنامه از سه بخش تشکیل شده است:
\begin{itemize}
\item 
\lr{Registeration}
\\
در این قسمت
\lr{Username}
و
\lr{Password}
و
\lr{Email}
توسط 
\lr{Query}
های 
\lr{URL}
از کاربر گرفته می شود. و در صورت وجود نداشتن 
\lr{Username}
و
\lr{Email}
کاربر به 
\lr{Database}
اضافه شده و 
\lr{Verification Code}
برای کاربر ارسال می شود.
\item \lr{Verification}
\\
در این قسمت
\lr{Username}
و
\lr{Verification Code}
توسط 
\lr{Query}
های 
\lr{URL}
از کاربر گرفته می شود. و در صورت وجود 
\lr{Username}
و همخوانی کد فعال سازی اکانت کاربر فعال می شود.
برای کاربر ارسال می شود.
\item \lr{Handshaking}
\\
در این قسمت سرور در صورت دریافت کلمه قرار دادی 
\lr{Hello}
پاسخ 
\lr{Hello back}
را برای کاربر ارسال می کند تا دو طرف متوجه برقراری ارتباط با یکدیگر شوند.
\end{itemize}
\item{\lr{\textbf{Login Application}}}
\\
این برنامه از دو قسمت تشکیل شده است:
\begin{itemize}
\item \lr{Login}
\\
در این قسمت
\lr{Username}
و
\lr{Password}
توسط 
\lr{Query}
های 
\lr{URL}
از کاربر گرفته می شود. و در صورت موجود بودن آن ها کاربر اجازه دسترسی به حساب کاربری خود را پیدا می کند.

\item \lr{Forget}
\\
در این قسمت
\lr{Email}
توسط 
\lr{Query}
های 
\lr{URL}
از کاربر گرفته می شود. و در صورت موجود بودن آن
\lr{Usename}
و
\lr{Password}
برای کاربر ارسال می شود.
\end{itemize}
\item \lr{Database}
\\
سرور
\lr{Zprava}
از ابزار 
\lr{SQLite3}
برای ذخیره اطلاعات استفاده میکند.
\\
به عنوان مثال هر کاربر اضافه شده به پیامرسان دارای پنج ویژگی در 
\lr{Database}
خواهد بود. این اطلاعات شامل
\lr{Username}
و
\lr{Password}
و
\lr{Email}
و
\lr{Verification Code}
و متغیر بولی 
\lr{is verified}
برای چک کردن کد تاییدیه است.


\end{enumerate}


\end{document}